\chapter{Equation 677}\label{677-chapter}

In this chapter we study finite magmas that obey equation 677,
\begin{equation}\label{677}
  x = y \op (x \op ((y \op x) \op y))
\end{equation}
for all $x,y$, and whether this implies equation 255,
\begin{equation}\label{255}
  x = ((x \op x) \op x) \op x.
\end{equation}
Using the usual notation $L_y x = y \op x$, $R_y x = x \op y$, $Sx = x \op x$, we can rewrite equation 677 as
\begin{equation}\label{677-alt}
  x = L_y L_x L_{L_y x} y
\end{equation}
and 255 as
$$ x = (Sx \op x) \op x.$$

\begin{lemma}[Basic properties of 677 magma]\label{677-basic} Let $M$ be a finite magma obeying \eqref{677}.
  \begin{itemize}
  \item (i)  The left multiplication operators $L_y: M \to M$ are all invertible.
  \item (ii) If $x,y \in M$ and $y \op x = x$, then $x = Sx \op x$.  In particular, 255 holds if and only if the equation $y \op x = x$ is solvable for every $x$.
\end{itemize}
\end{lemma}

\begin{proof}  From \eqref{677-alt} we see that $L_y$ is surjective, hence invertible on finite magmas, giving (i).  For (ii), we apply \eqref{677} to conclude that
  $$ y \op x = x = y \op (x \op (x \op y))$$
  and hence by left invertibility
  $$ x = x \op (x \op y).$$
  On the other hand, from \eqref{677} with $y$ replaced by $x$ we have
  $$ x = x \op (x \op (Sx \op x))$$
  and the claim then follows by left invertibility.
\end{proof}

This for instance gives the implication for linear magmas:

\begin{lemma}[No linear counterexamples]\label{linear-obstruction}  Suppose we have a finite magma $M$ obeying 677 which is linear in the sense that $M$ is an abelian group and $x \op y = \alpha x + \beta y + c$ for some endomorhpisms $\alpha,\beta: M \to M$ and constant $c$.  Then $M$ obeys 255.
\end{lemma}

\begin{proof}  By the previous lemma, it suffices to show that right multiplication $R_x$ is surjective, or equivalently injective by finiteness.  If this is not the case, then we can find distinct $y,y'$ such that $R_x y = R_x y'$, hence $L_y x = L_{y'} x$. But in this linear model, $L_y$ and $L_{y'}$ differ by a constant, hence we have $L_y = L_{y'}$.  Applying \eqref{677-alt} we have
$$  L_y L_x L_{L_y x} y = x =  L_y L_x L_{L_y x} y'$$
and hence by left-invertiblity $y=y'$, a contradiction.
\end{proof}

In fact the argument gives a stronger obstruction to refuting 255:

\begin{lemma}[No counterexamples via linear extension]\label{linear-2} Suppose that we have a magma with carrier $G \times M$ obeying 677, where $G$ already is a magma obeying 677 and 255, $M$ is an abelian group, and the multiplication operation on $G \times M$ is of the form
  $$ (x,s) \op (y,t) = (x \op y, \alpha_{x,y} s + \beta_{x,y} t + c_{x,y})$$
for some endomorphisms $\alpha_{x,y},\beta_{x,y}: M \to M$ and constants $c_{x,y}$.  Then $G \times M$ obeys 255.
\end{lemma}

\begin{proof}  By \Cref{677-basic}, it suffices to show that for any $(y,t)$, the equation $(x,s) \op (y,t) = (y,t)$ is solvable.  Since $G$ already obeys 255, we know that we can find $x$ such that $x \op y = y$, so it suffices to show that the operation $s \mapsto \alpha_{x,y} s + \beta_{x,y} t + c_{x,y}$ is surjective, or equivalently injective.  If this were not the case, then we could find $s,s'$ such that $\alpha_{x,y} s = \alpha_{x,y} s'$, and hence $L_{(x,s)} (y,t') = L_{(x,s')} (y,t')$ for all $t' \in M$.  Since
  $$ x \op y = y = x \op (y \op ((x \op y) \op x))$$
  from \eqref{677}, we have $y = y \op ((x \op y) \op x)$, and hence $L_{(y,t)} L_{L_{(x,s)} (y,t)} (x,s)$ is of the form $(y,t')$ for some $t'$, and similarly with $(x,s)$ replaced by $(x,s')$.  We conclude that
$$ L_{(x,s)} L_{(y,t)} L_{L_{(x,s)} (y,t)} (x,s) = (y,t) =
L_{(x,s)} L_{(y,t)} L_{L_{(x,s)} (y,t)} (x,s')$$
and hence by left invertibliity $s=s'$, a contradiction.
\end{proof}

Linear models $x \op y = \alpha x + \beta y + c$ on the finite field $F_p$ turn out to be classified into two types:
\begin{itemize}
\item (Type 1) $\alpha=1-\beta$, $\beta$ is a primitive tenth root of unity, and $c=0$.  (These models are also translation-invariant.)
\item (Type 2) $\alpha$ is a primitive third root of unity, $c$ is arbitrary, and $\beta$ solves $\beta^3+\beta+1=-\alpha$ and $\beta^4+\beta^3+2\beta^2+2\beta+1=0$.
\end{itemize}
An example of a Type I model is $x \op y = 2x-y$ on $F_5$.  Examples of Type II models include $x \op y = 4x+3y$ and $x \op y = 4x+y$ on $F_7$.  An exceptional class of Type II models are $x \op y = 5x-4y+c$ on $F_{31}$, these are the only Type II models that are translation-invariant and do not have idempotents (if $c \neq 0$).

\section{The free 677 magma}

In this section we construct the free 677 magma $M_{X,677}$ generated by some set of generators $X$.  First let $M_X$ be the free magma generated $X$ with operation given by pairing $x,y \mapsto (x,y)$; one can think of elements of $M_X$ as finite trees with leaves in $X$.  If $w = (x,y)$ we write $x = w_L$ and $y = w_R$ for the left and right components of $w$; we also define $w_{LL}$, $w_{LR}$, etc. iteratively if they are defined, for instance if $w = ((x,y),z)$ then $w_{LR} = y$.  We define a partial order $<$ on $M_X$ by declaring $w < w'$ if $w$ is a subtree of $w'$, thus $w < w'$ if one of $w \leq w'_L$, $w \leq w'_R$ holds.

We define an operation $\diamond$ recursively on $M_X$ by the following rule:
\begin{itemize}
\item If $x,y \in M_X$ is such that $x < y = (y_L, (x \diamond y_L) \diamond x)$, then $x \diamond y := y_L$.  Otherwise, $x \diamond y = (x,y)$.
\end{itemize}

Note that to define $x \diamond y$, one only needs to be able to compute $x' \diamond y'$ for $y' < y$.  Since there are no infinite descending chains in the partial order $<$, we see that $\diamond$ is well-defined.  By construction, we observe the following properties:

\begin{lemma}[Properties of operation]\label{op-prop}  Let $x,y \in M_X$ be such that $x \diamond y = z$.  Then either
$$ x, y < z = (x,y)$$
or
$$ x, z < y = (z, (x \diamond z) \diamond x).$$
In particular, $x$ is strictly upper bounded by one of $y,z$.
\end{lemma}

Next, we observe

\begin{lemma}[Additional property]\label{op-2}  If $x,y \in M_X$, then
  $$ x \op ((y \op x) \op y) = (x, (y \op x) \op y).$$
\end{lemma}

\begin{proof} Write $z := y \op x$, $u = z \op y$, $v = x \op u$.  Our task is to show that $v = (x,u)$.

From \Cref{op-prop} we know that $y$ is upper bounded by one of $z,x$, $z$ is upper bounded by one of $u,y$, and $x$ is upper bounded by one of $u,v$.  So out of $x,y,z,u,v$, the only elements that can be maximal in this set are $u$ and $v$.  If $v$ is maximal, then by \Cref{op-prop} we have $v = (x,u)$ as required, hence we may assume for contradiction that $u$ is maximal.  From \Cref{op-prop}, this implies that $u = (z,y)$ and $u = (u_L, (x \op u_L) \op x)$, hence $u_L = z$ and $y = (x \op z) \op x$.

From \Cref{op-prop}, $x$ is upper bounded by one of $x \op z$ and $z$, and $x \op z$ is upper bounded by one of $y$ and $x$.  We also recall that $y$ was upper bounded by one of $z,x$. We conclude that out of $x,y,z,x \op z$, the only one that can be maximal is $z$.  In particular $z$ is not bounded by $x$, hence by \Cref{op-prop} $z = (y,x)$.  $z$ is also not bounded by $w$, hence by \Cref{op-prop} $x \op z = z_L = y$, hence $y = y \op x = z$, contradicting the fact that $y$ was not maximal. This gives the required contradiction.
\end{proof}

\begin{corollary}\label{677-obey}  The operation $\diamond$ obeys \eqref{677}.
\end{corollary}

\begin{proof} By the previous lemma and definition of $\diamond$, it suffices to show that $y < (x, (y \op x) \op y)$.  Defining $z,y,v$ as before, this amounts to showing that $y < v$.  We already have $v = (x,u)$, hence by \Cref{op-prop} $x < u$.  Also recall that $y$ is bounded by one of $x,z$, and $z$ is bounded by one of $y,u$.  Since $u$ is also bounded by $v$, we obtain the claim.
\end{proof}

\begin{corollary} Let $M_{X,677}$ be the magma generated by $X$ with operation $\diamond$.  Then $M_{X,677}$ is the free magma for \eqref{677} generated by $X$.
\end{corollary}

\begin{proof}  By the previous corollary, it suffices to show that every function $f: X \to M$ into a 677 magma $M$ can be extended to a unique homomorphism $\varphi_f: M_{X,677} \to X$. Uniqueness is clear since $M_{X,677}$ is generated by $X$.  For existence, we define $\varphi_f$ by first extending $f$ to the unique homomorphism from $M_X$ to $X$ (using the pairing map) and then restricting to $M_{X,677}$.  To verify the homomorphism property $\varphi_f(x \diamond y) = \varphi_f(x) \diamond \varphi_f(y)$, we are already done when $x \diamond y = (x,y)$. The only remaining case is when $x \diamond y = y_L$ and $x < y = (y_L,  (x \diamond y_L) \diamond x)$.   If we assume inductively that the homomorphism property $\varphi_f(x' \diamond y') = \varphi_f(x') \diamond \varphi_f(y')$ has already been verified for $y' < y$, then we have
  $$ \varphi_f(y) = \varphi_f(y_L) \diamond \varphi_f(y_R) = \varphi_f(y_L) \diamond (\varphi_f(x \diamond y_L) \diamond \varphi_f(x)) =  \varphi_f(y_L) \diamond ((\varphi_f(x) \diamond \varphi_f(y_L)) \diamond \varphi_f(x))$$
  and the claim now follows since $M$ obeys 677.
\end{proof}

By construction, we have $x \op y > y$ or $x \op y < y$ for any $x,y$.  In particular, $M_{X,677}$ does not obey \eqref{255}.
