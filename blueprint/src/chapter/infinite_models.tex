\chapter{Infinite models}\label{infinite-model-chapter}

In this chapter we consider non-implications which are refuted only on infinite models, as those are
more challenging to prove---they can't be proved by directly giving an operation table and checking
which laws it satisfies.

The singleton or empty magma obeys all equational laws.  One can ask whether an equational law admits nontrivial finite or infinite models.  An \emph{Austin law} is a law which admits infinite models, but no nontrivial finite models.  Austin \cite{austin} established the first such law, namely the order $9$ law
$$ (((1 \op 1) \op 1) \op 0) \op (((1 \op 1) \op ((1 \op 1) \op 1)) \op 2) \formaleq 0.$$
A shorter Austin law of order $6$ was established in \cite{Kisielewicz}:

\begin{theorem}[Kisielewicz's first Austin law]
  \lean{InfModel.Finite.Equation374794_implies_Equation2,InfModel.Equation374794_not_implies_Equation2}\leanok
  \label{kis-thm}\uses{eq374794,eq2}
  \Cref{eq374794} is an Austin law.
\end{theorem}

\begin{proof} \leanok  First we show that every finite model of \Cref{eq374794} is trivial.  Write $y^2 := y \op y$ and $y^3 := y^2 \op y$.  For any $y,z$, introduce the functions ${f_y: x \mapsto y^3 \op x}$ and ${g_{yz}: x \mapsto x \op (y^2 \op z)}$.  \Cref{eq374794} says that $g_{yz}(f_y(x))=x$, hence by finiteness $g_{yz}=f_y^{-1}$, showing that $g_{yz}$ does not depend on the value of $z$.  Since
$$ f_y(y^2 \op z) = g_{yz}(y^3),$$
it follows that $f_y(y^2 \op z)=f_y(y^3)$ which by injectivity of $f_y$ implies that $z\mapsto y^2 \op z$ is a constant function (with $y$ fixed).  Substituting $y^2$ for $y$ shows that the same is true for $z\mapsto (y^2 \op y^2) \op z$, and since
$$ f_y(z) = (y^2 \op y) \op z = (y^2 \op y^2) \op z$$
we conclude that $f_y$ is also a constant function.  But this function is already known to be injective, thus there do not exist distinct elements in its domain, showing that the model must be trivial.

To construct an infinite model, consider the magma of positive integers $\Z^+$ with the operation $x \op y$ defined by
$$
x \op y =
\begin{cases}
2^y, & x=  y            \\
3^y, & x=  1,\ y \neq 1 \\
  z, & x=3^z,\ y \neq x \\
  1, & else
\end{cases}.
$$
Then $y \op y = 2^y$ and $(y \op y) \op y = 1$ for all $y$.  If $x\neq 1$ we have that
$$ ((y \op y) \op y) \op x = 3^x, $$
and since $(y \op y) \op z$ is a power of two for all $y, z$ it follows that
$$ 3^x \op ((y \op y) \op z) = x. $$
The case $x=1$ requires a further argument: observe that $w = (y \op y) \op z$ evaluates to one unless $z = 2^y$, in which case it evaluates to $2^{2^y}$ (which is greater than or equal to four).  In particular, $w$ never takes the value two.  Thus
$$ (((y \op y) \op y) \op 1) \op ((y \op y) \op z) = 2 \op w = 1, $$
concluding our proof that this magma is a model of \Cref{eq374794}
\end{proof}

An even shorter law (order $5$) was obtained by the same author in a follow-up paper \cite{Kisielewicz2}:

\begin{theorem}[Kisielewicz's second Austin law]\label{kis-thm2}\uses{eq2, eq28770}\lean{InfModel.Finite.Equation28770_implies_Equation2}\lean{InfModel.Equation28770_not_implies_Equation2}\leanok \Cref{eq28770} is an Austin law.
\end{theorem}

\begin{proof} \leanok Using the $y^2$ and $y^3$ notation as before, the law reads
\begin{equation}\label{kis2-law}
   x = (y^3 \op x) \op (y \op z).
  \end{equation}
In particular, for any $y$, the map $T_y \colon x \mapsto y^3 \op x$ is injective, hence bijective in a finite model $G$.  In particular we can find a function $f : G \to G$ such that $T_y f(y) = y^3$ for all $y$  Applying \Cref{kis2-law} with $x = f(y)$, we conclude
$$ T_y(y \op z) = y^3 \op (y \op z) = f(y) $$
and thus $y \op z$ is independent of $z$ by injectivity of $T_y$.  Thus, the left-hand side of \Cref{kis2-law} does not depend on $x$, and so the model is trivial.  This shows there are no non-trivial finite models.

To establish an infinite model, use $\N$ with $x \op y$ defined by requiring
$$ y \op y = 2^y; \quad 2^y \op y = 3^y$$
and
$$ 3^y \op x = 3^y 5^x$$
for $x \neq 3^y$, and
$$ (3^y 5^x) \op z = x$$
for $z \neq 3^y 5^x$.  Finally set
$$ 2^{3^y} \op z = 3^y$$
for $z \neq 3^y, 2^{3^y}$.  All other assignments of $\op$ may be made arbitrarily. It is then a routine matter to establish \Cref{kis2-law}.
\end{proof}

In that paper a computer search was also used to show that no law of order four or less is an Austin law.

An open question is whether \Cref{eq5093} is an Austin law.  We have the following partial result from \cite{Kisielewicz2}:

\begin{theorem}[Equation 5093 has no non-trivial finite models]\lean{InfModel.Finite.Equation5093_implies_Equation2}\leanok\label{5093-nontrivial}\uses{eq5093, models-def} \Cref{eq5093} has no non-trivial finite models.
\end{theorem}

\begin{proof} \leanok From \Cref{eq5093} we see that the map $w \mapsto y \op w$ is onto, hence injective in a finite model.  Using this injectivity four times in \Cref{eq5093}, we see that $z \op y$ does not depend on $z$, hence the expression
$x \op (z \op y)$ does not depend on $x$.  By \Cref{eq5093} again, this means that $x$ does not depend on $x$, which is absurd in a non-trivial model.
\end{proof}

We also have such a non-implication involving two laws of order $4$:

\begin{theorem}[3994 implies 3588 for finite models]\label{finite_imp_3994_3588_thm}\uses{eq3994,eq3588,models-def}
  \lean{InfModel.Finite.Equation3994_implies_Equation3588}\leanok
  All finite magmas which satisfy \Cref{eq3994} also satisfy \Cref{eq3588}.
\end{theorem}

\begin{proof}\label{finite_imp_3994_3588} \leanok
  For a finite magma $M$, consider the set $S = \{x \op y | x, y \in M\}$.
  Now $f_z : x \mapsto z \op x$ and $g_z : x \mapsto x \op z$.
  They both map $S$ to $S$, and due to the hypothesis $g_z \op f_z$ is the identity on $S$,
  so because $S$ is finite $f_z$ and $g_z$ must be inverse bijections on it, and therefore
  they commute.
\end{proof}

\begin{theorem}[3994 does not imply 3588 for infinite models]\label{non_imp_3994_3588_thm}\uses{eq3994,eq3588,models-def}
  \lean{InfModel.Equation3994_not_implies_Equation3588}\leanok
  There exists a magma which satisfies \Cref{eq3994} and not \Cref{eq3588}.
\end{theorem}

\begin{proof} \leanok
  Consider $\mathbb{N}$, with $x \op y$ defined as $x \oplus y$ (bitwise XOR) if $x$ and $y$ are even,
  $y+2$ if only $y$ is even, $x \dot - 2$ if only $x$ is even, and $0$ if both are odd.
  Note that the range of the operation is the set of even naturals.
  \Cref{eq3994} is satisfied, because for even $z$ we get $z \oplus (x \op y) \oplus z = x \op y$
  and for odd $z$ we get $(x \op y) + 2 \dot - 2 = x \op y$.
  Setting $x = y = z = 1$, \Cref{eq3588} isn't satisfied.
\end{proof}

The following result was established in \cite{austin_finite}:

\begin{theorem}[Austin's finite model theorem]\label{austin-two}\uses{models-def}\lean{InfModel.Finite.two_variable_laws}\leanok Any law with at most two variables has a non-trivial finite model.
\end{theorem}

\begin{proof} \leanok  If neither side of the law is a single variable then the zero law $x \op y = 0$ will work, so one can assume the law takes the form $x = f(x,y)$.  Consider a finite field $F$ with the operation $x \op y := ax+by$ for some coefficients $a,b \in F$.  Then the law becomes a pair of equations $P(a,b)=0$, $Q(a,b)=1$ in the coefficients for some polynomials $P,Q$ with integer coefficients, which one can verify to not divide each other (they have the same degree, and do not have the same set of non-zero monomials).  From Bezout's theorem, this equation has a solution in some field, and hence by the Lefschetz principle it has a solution in a finite field.
\end{proof}

Many implications for finite magmas rely on the fact that if $f, g:X \to X$ are functions on a finite set $X$, then $f \circ g = I$ if and only if $g \circ f = I$.  Two more complicated variants of this are as follows.

\begin{lemma}\label{ffg} \lean{FiniteModel.Finite.f_ffg_implies_f_fgf} \leanok Let $X$ be finite, and let $f, g: X \to X$ be such that $f = f \circ f \circ g$.  Then $f = f \circ g \circ f$.
\end{lemma}

\begin{proof} \leanok Call a point $x \in X$ \emph{periodic} if $f^n(x) = x$ for some $n>0$.  Because the forward orbit $x, f(x), f^2(x), \dots$ in a finite space $X$ must repeat, we see that for each $x$ there exists an $n$ such that $f^n(x)$ is periodic.  Let $n_x$ be the first $n$ for which this occurs.  If $n_x > 1$, then $g(x), f(g(x)), f(f(g(x))$ cannot be periodic (as this would imply $f(x)$ periodic), and then we can see that $n_{g(x)} = n_x + 1$.  Hence the maximal value of $n_x$ is at most $1$, which implies that $f(x)$ is periodic for every $x$.  Thus there is an $n > 1$ such that $f^n = f$.  Since $f = f^2 \circ g$, we have $f^n = f^{n-2} \circ f \circ f = f^{n-2} \circ (f^2 \circ g) \circ f = f^n \circ g \circ f = f \circ g \circ f$, as desired.
\end{proof}

\begin{lemma}\label{gff} \lean{FiniteModel.Finite.f_gff_implies_f_fgf} \leanok Let $X$ be finite, and let $f, g: X \to X$ be such that $f = g \circ f \circ f$.  Then $f = f \circ g \circ f$.
\end{lemma}

\begin{proof} \leanok By hypothesis, $f^2(x)$ uniquely determines $f(x)$.  This prevents $n_x = 2$ for any $x$ (because if $n>2$ is such that $f^{n+2}(x) = f^2(x)$, then $f^{n+1}(x) \neq f(x)$), and hence also prevents $n_x > 2$ for any $x$.   So again we have $f(x)$ periodic for all $x$, so $f^n = f$ for some $n>1$.  Then $f = f^n = f \circ (g \circ f^2) \circ f^{n-2} = f \circ g \circ f^n = f \circ g \circ f$ as required.
\end{proof}

This can be used to obtain a few positive finite magma implications, for instance by setting $f,g$ to be left and right multiplication operators.  Another useful lemma is

\begin{lemma}[Eventual period]\label{period} \lean{FiniteModel.Finite.fn_periodic} \leanok  Let $X$ be finite and $f: X \to X$.  Then there exists $n \geq 1$ such that $f^{2n} = f^n$.
\end{lemma}

\begin{proof} \leanok By the pigeonhole principle, there exists $n \geq 1$, $m \geq 0$ such that $f^{m+n} = f^m$, which implies on iteration that $f^{m+mn} = f^m$ and hence $f^{2mn} = f^{mn}$, giving the claim.
\end{proof}

As a sample application, we have

\begin{corollary}[3342]\label{3342} \lean{Eq3342.Finite.Equation3342_implies_Equation3522,Eq3342.Finite.Equation3342_implies_Equation4118} \leanok  On a finite magma $M$, equation 3342, $x \op y= y \op (x \op (x \op x))$, implies
  equation 3522, $x \op y = x \op ((y \op y) \op y)$, as well as equation 4118, $x \op y = ((x \op x) \op x) \op y$.
\end{corollary}

\begin{proof}\uses{period} \leanok Write $Sx := x \op$, $fx := x \op Sx$ and $Cx = Sx \op x$, then we have $x \op y= y \op fx$, and our task is to show $x \op y = Cx \op y$ and $x \op y = x \op Cy$ for finite magmas, giving the four open implications.

Note that $x \op y= y \op fx = fx \op fy$, hence $f$ is a homomorphism.  By \Cref{period}, there is $n \geq 1$ with $f^n = f^{2n}$.  From 3342 we have $Sx = S fx$ and $Cx = Sx \op x = f^n Sx \op f^n x = f^n Sx \op f^{2n} x = f^{2n-1} x \op f^n Sx = f^{n-1} x \op S x = f(f^{n-1} x) = f^n x$.  Then
$$ Cx \op y = f^n x \op y = f^{2n} x \op f^n y = f^n x \op f^n y = x \op y$$
and a similar argument gives $x \op Cy = x \op y$.
\end{proof}
