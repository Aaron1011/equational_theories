\section{Counterexample constructions}

\note{TODO: Expand this sketch}

\subsection{Finite magmas}\label{finite-sec}

Discuss semi-automated creation of finite counterexamples (as discussed here)
Describe various sources of example magmas used in counterexamples, including the ones listed here.

Also note some "negative results" - classes of finite magmas that did not yield many additional refutations, e.g. commutative 5x5 magmas.

Mention finite immunity

Using SAT solvers to find medium sized finite magmas obeying a given law? See this discussion.

Discuss computational and memory efficiencies needed to brute force over extremely large sets of magmas. SAT solving may be a better approach past a certain size!

\subsection{Linear models}\label{linear-sec}

Note both commutative and noncommutative models

Mention Lefschetz principle - commutative models can be made finite

Note that linear magmas let one assign an "affine scheme" to each law that can be used to rule out many, but not all, implications.

Mention Linear immunity



\subsection{Translation-invariant models}\label{translation-sec}

Translation-invariant magmas (see e.g., this thread for a nice example). Note: any magma with a transitive symmetry will lift to a translation-invariant model, so this helps explain why these are common examples. Also symmetric models could be slightly more likely to obey various laws than general models due to degree of freedom considerations.

Mention translation-invariant immunity

\subsection{Ad hoc constructions}\label{adhoc-sec}

Tree based constructions, see here.

\subsection{Greedy constructions}\label{greedy-sec}


\subsection{Modifying base models}\label{modify-base}
