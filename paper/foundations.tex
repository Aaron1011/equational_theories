\section{Notation and Mathematical Foundations}\label{notation-sec}

If $\Magma = (M,\op)$ is a magma, we define the left and right multiplication operators $L_a, R_a \colon M \to M$ for $a \in M$ by the formula
\begin{equation}\label{left-right}
    L_x y = R_y x \coloneqq x \op y.
\end{equation}
We also define the squaring operator $S \colon M \to M$ by
\begin{equation}\label{square-def}
    Sx \coloneqq x \op x = L_x x = R_x x.
\end{equation}
%and the (right) cubing operator $C \colon M \to M$ by
%\begin{equation}\label{cube-def}
%    Cx \coloneqq Sx \op x = R_x^2 x.
%\end{equation}

A \emph{homomorphism} $f \colon \Magma \to \Magma'$ between two magmas $\Magma = (M,\op)$, $\Magma' = (M',\op')$ is a function $f \colon M \to M'$ such that $f(x \op y) = f(x) \op' f(y)$ for all $x,y \in M$.  An \emph{isomorphism} is a homomorphism that is invertible (which implies that the inverse is also a homomorphism).  An \emph{endomorphism} is a homomorphism from a magma to itself.

If $X$ is an alphabet, we let $\Magma_X$ denote the free magma generated by $X$, thus an element of $\Magma_X$ is either a letter in $X$, or of the form $w_1 \op w_2$ with $w_1,w_2 \in \Magma_X$.  Every function $f \colon X \to M$ into a magma $\Magma = (M,\op)$ extends to a unique homomorphism $\varphi_f \colon \Magma_X \to \Magma$.  Formally, an equational law with some indeterminates in $X$ can be written as $w_1 \formaleq w_2$ for some $w_1, w_2 \in X$; a magma $\Magma = (M,\op)$ then obeys this law if and only if $\varphi_f(w_1) = \varphi_f(w_2)$ for all $f \colon X \to M$.  We also define the order of a word $w \in \Magma_X$ to be the number of occurrences of $\op$ in the word, thus letters in $X$ are of order $0$, and the order of $w_1 \op w_2$ is the sum of the orders of $w_1, w_2$.

A \emph{theory} is a collection $\Gamma$ of equational laws; we say that a magma $\Magma$ \emph{satisfies} a theory, and write $\Magma \models \Gamma$, if every law in $\Gamma$ is obeyed by $\Magma$.  If $E$ is an equational law, we write $\Gamma \vdash E$ if every magma that satisfies $\Gamma$ also satisfies $E$. A \emph{free magma} $\Magma_{X,\Gamma}$ for such a theory $\Gamma$ and an alphabet $X$ is a magma satisfying $\Gamma$ together with a map $\iota_{X,\Gamma} \colon X \to \Magma_{X,\Gamma}$ which is universal in the sense that every function $f \colon X \to \Magma$ to a magma $\Magma$ satisfying $\Gamma$ uniquely determines a homomorphism $\varphi_{f,\Gamma} \colon \Magma_{X,\Gamma} \to \Magma$ such that $\phi_{f,\Gamma} \circ \iota_{X,\Gamma} = f$.  This magma is unique up to isomorphism; a canonical way to construct it is as the quotient $\Magma_X/\sim_\Gamma$ of the free magma $\Magma_X$ by the equivalence relation $\sim_\Gamma$ given by declaring $w \sim_\Gamma w'$ if $\Gamma \vdash w \formaleq w'$.  If $\Gamma = \{E\}$ consists of a single law $E$, we write $\Magma_{X,E}$, $\sim_E$, $\varphi_{f,E}$ for $\Magma_{X,\{E\}}$, $\sim_{\{E\}}$, $\varphi_{f,\{E\}}$ respectively . \note{Give reference for free magmas relative to theories}

In general, the free magma $\Magma_{X,\Gamma}$ is difficult to describe in a tractable form, but for some theories, one has a simple description:

\begin{example}[Commutative and associative free magma]\label{semi-group} The free magma $\Magma_{X,\{E43, E4512\}}$ for the commutative law \eqref{eq43} and the associative law \eqref{eq4512} is the free abelian semigroup generated by $X$ (with $\iota_{X,\{E43,E4512\}}$ the obvious embedding map).
\end{example}

\begin{example}[Left-absorptive free magma]\label{left-absorb}
The free magma $\Magma_{X,\{E4\}}$ for the left-absorptive law \eqref{eq4} is the magma with carrier $X$ and operation $x \op y = x$ (with $\iota_{X,E_4}$ the identity).
\end{example}


Every magma $\Magma$ has an opposite $\Magma^{\mathrm{op}}$, which has the same carrier but the opposite operation $x \op^{\mathrm{op}} y \coloneqq y \op x$.  A magma $\Magma$ obeys an equational law $E$ if and only if its opposite $\Magma^{\mathrm{op}}$ obeys the dual law $E^*$, defined by reversing the all operations.  For instance, the dual of
$\x \op \y \formaleq \x \op (\y \op \z)$ \eqref{eq327} is $\y \op \x \formaleq (\z \op \y) \op \x$, which in our numbering system we rewrite in normal form as $\x \op \y \formaleq (\z \op \x) \op \y$ \eqref{eq395}.

We then see that the implication graph has a duality symmetry: given two equational laws $E_1,E_2$, we have $E_1 \vdash E_2$ if and only if $E_1^* \vdash E_2^*$.

\section{Formal Foundations}

\note{TODO: expand this sketch.}

Here we describe the Lean framework used to formalize the project, covering technical issues such as:

\begin{itemize}
    \item Magma operation symbol issues
    \item Syntax (`LawX`) versus semantics (`EquationX`)
    \item "Universe hell" issues
    \item Additional verification (axiom checking, Leanchecker, etc.)
    \item Use of the `conjecture` keyword
    \item Use of namespaces to avoid collisions between contributions. (Note: we messed up with this with FreeMagma! Should have namespaced back end results as well as front end ones.)
    \item Use of Facts command to efficiently handle large numbers of anti-implications at once
\end{itemize}

Upstream contributions:

\begin{itemize}
    \item \href{https://github.com/leanprover-community/mathlib4/pulls?q=is%3Apr+is%3Abody+EquationalTheories+}{Mathlib contributions}
    \item \href{https://github.com/PatrickMassot/leanblueprint/pulls?q=is%3Apr+in%3Abody+EquationalTheories+}{LeanBlueprint contributions}
\end{itemize}
