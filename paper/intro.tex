\section{Introduction}

The purpose of this paper is to report on the \emph{Equational Theories Project} (ETP)\footnote{\url{https://teorth.github.io/equational_theories/}}, a pilot project launched\footnote{\url{https://terrytao.wordpress.com/2024/09/25}} in September 2024 to explore new ways to collaboratively work on mathematical research projects using machine assistance. The project goal, in the area of universal algebra, was selected to be particularly amenable to crowdsourced and computer-assisted techniques, while still being of mathematical research interest. \note{Describe outcomes}

\subsection{Magmas and Equational Laws}

In order to describe the mathematical goals of the ETP, we need some notation. A \emph{magma} $M = (M,\op)$ is a set $M$ (known as the \emph{carrier}) together with a binary operation $\op \colon M \times M \to M$. An \emph{equational law} for a magma, or \emph{law} for short, is an identity involving $\op$ and some indeterminates, which we will typically denote using the symbols $x,y,z,u,v,w$. Familiar examples of equational laws include the \emph{commutative law}
\begin{equation}\label{eq43}\tag{E43}
    x \op y = y \op x
\end{equation}
and the \emph{associative law}
\begin{equation}\label{eq4512}\tag{E4512}
    (x \op y) \op z = x \op (y \op z).
\end{equation}
In the ETP, a unique number was assigned to each equational law, via a numbering system that we describe in \Cref{numbering-app}. Formally, one can represent an equational law syntactically as a string $w_1 \formaleq w_2$, where $w_1, w_2$ are words in a free magma generated by formal indeterminate symbols; see ???.

A magma $M$ obeys a law $E$ if the law $E$ holds for all possible assignments of the indeterminate to $M$, in which case we write $M \models E$. Thus for instance $M \models E43$ if one has $x \op y = y \op x$ for all $x,y \in M$.

We place a pre-order on laws by writing $E \leq E'$ or $E \vdash E'$ if every magma that obeys $E$, also implies $E'$: $(M \models E) \implies (M \models E')$. We say that two laws are \emph{equivalent} if they imply each other. For instance, the constant law
\begin{equation}\label{eq46}\tag{E46}
x \op y = z \op w
\end{equation}
can easily be seen to be equivalent to the law
\begin{equation}\label{eq41}\tag{E41}
x \op x = y \op z.
\end{equation}
In this pre-ordering, a maximal element is given by the trivial law
\begin{equation}\label{eq1}\tag{E1}
x = x
\end{equation}
and a minimal element is given by the singleton law
\begin{equation}\label{eq2}\tag{E2}
x = y.
\end{equation}

The \emph{order} of an equational law is the number of occurrences of the magma operation. For instance, the commutative law \eqref{eq43} has order $2$, while the associative law \eqref{eq4512} has order $4$. We note some selected laws of small order that have previously appeared in the literature:
\begin{itemize}
\item The \emph{central groupoid law}
\begin{equation}\label{eq168}\tag{E168}
x = (y \op x) \op (x \op z)
\end{equation}
is an order $3$ law introduced by Evans \cite{evans} and studied further by Knuth \cite{knuth} and many further authors, being closely related to central digraphs (also known as unique path property diagraphs), and leading in particular to the discovery of the Knuth-Bendix algorithm \cite{knuth-bendix}; see \cite{klt} for a more recent survey.
\item \emph{Tarski's axiom}
\begin{equation}\label{eq543}\tag{E543}
    x = y \op ( (z \op (x \op (y \op z))))
\end{equation}
is an order $4$ law that was shown by Tarski \cite{Tarski1938} to characterize the operation of subtraction in an abelian group; that is to say, a magma $M$ obeys \eqref{eq543} if and only if there is an abelian group structure on $M$ for which $x \op y = x-y$ for all $x,y \in M$.
\item In a similar vein, it was shown in \cite{mendelsohn-padmanabhan} (see also \cite{meredith-prior}) that the order $4$ law
\begin{equation}\label{eq1571}\tag{E1571}
    x = (y \op z) \op (y \op (x \op z))
\end{equation}
characterizes addition (or subtraction) in an abelian group of exponent $2$; it was shown in \cite{mccune_et_al} that the order $4$ law
\begin{equation}\label{eq345169}\tag{E345169}
    x = (y \op ((x \op y) \op y)) \op (x \op (z \op y))
\end{equation}
characterizes the Sheffer stroke in a boolean algebra, and it was shown in \cite{higman-neumann} that the order $8$ law
\begin{equation}\label{eq42323216}\tag{E42323216}
x = y \op ((((y \op y) \op x) \op z) \op (((y \op y) \op y) \op z))
\end{equation}
characterizes division in a (not necessarily abelian) group.
\end{itemize}
Some further examples of laws characterizing well-known algebraic structures are listed in \cite{mccune-survey}.

The Birkhoff completeness theorem \cite[Th. 3.5.14]{term-rewriting} implies that an implication $E \vdash E'$ of equational laws holds if and only if the left-hand side of $E'$ can be transformed into the right-hand side by a finite number of substitution rewrites using the law $E$. However, the problem of determining whether such an implication holds is undecidable in general \cite{mckenzie}. Even when the order is small, some implications\footnote{Another contemporaneous example of this phenomenon was the solution of the Robbins problem \cite{robbins}.} can require lengthy computer-assisted proofs; for instance, it was noted in \cite{Kisielewicz2} that the order $4$ law
\begin{equation}\label{eq1689}\tag{E1689}
    x = (y \op x) \op ((x \op z) \op z)
\end{equation}
was equivalent to the singleton law \eqref{eq2}, but all known proofs are computer-assisted.

\subsection{The Equational Theories Project}

As noted in \Cref{numbering-app}, there are $4694$ equational laws of order at most $4$. The primary mathematical goal of the ETP was to completely determine the implication pre-ordering $\leq$ for this set of laws. Such systematic determinations of implication graphs have been seen previously in the literature; for instance, in \cite{phillips-vojtechovsky}, the relations between $60$ identities of Bol-Moufang type were established, and in the blog post \cite[\S 17]{Wolfram_2022}, some initial steps towards generating this graph for the first hundred or so laws on our list were performed. However, to our knowledge, the ETP is the first project to study such implications at the scale of thousands of laws.

The ETP requires the determination of the truth or falsity of $4694^2 = 22033636$ implications; while one can use properties such as the transitivity of the pre-ordering to reduce the work somewhat, this is clearly a task that requires significant automation. It was also a project highly amenable to crowdsourcing, in which different participants could work on developing different techniques, each of which could be used to fill out a different part of the implication graph. In this respect, the project could be compared with a Polymath project \cite{Gowers2009}, which used online forums such as blogs and wikis to openly collaborate on a mathematical research problem. However, the Polymath model required human moderators to review and integrate the contributions of the participants, which clearly would not scale to the ETP which required the verification of over twenty million mathematical statements. Instead, the ETP was centered around a Github repository in which the formal mathematical contributions had to be entered in the proof assistant language \emph{Lean}, where they could be automatically verified. In this respect, the ETP was more similar to the recently concluded Busy Beaver Challenge\footnote{\url{https://bbchallenge.org/}}, which was a similarly crowdsourced project that computed the fifth Busy Beaver number $BB(5)$ to be $47176870$ through an analysis of $88664064$ Turing machines, with the halting analysis being verified in a variety of computer languages, with the final formal proof written in the proof assistant language \emph{Coq}. One of the aims of the ETP was to explore potential workflows for such collaborative, formally verified mathematical research projects that could serve as a model for future projects of this nature.

Secondary aims of the ETP included the possibility of discovering unusually interesting equational laws, or new experimental observations about such laws, that had not previously been noticed in the literature; and to develop benchmarks to assess the performance of automated theorem provers and other AI tools.

\subsection{Outcomes}

The ETP achieved its primary objective, with all of the implications formalized in the proof assistant language \emph{Lean}, and can be found on the ETP GitHub repository. The experience of running such a large collaborative research project introduced several challenges, which we report upon in \Cref{project-sec}. Also, a variety of methods with varying degrees of automation or computer-assistance had to be developed to resolve all the implications, which had quite a variety of difficulty levels.

Of the $22033636$ possible implications $E \vdash E'$, $8178279$ (or $37.12\%$) would end up being true. To establish such positive implications $E \vdash E'$, the main techniques used were as follows:

\begin{itemize}
    \item A very small number of positive implications were established and formalized by hand, mostly through direct rewriting of the laws; but this approach would not scale to the full project.
    \item Simple rewriting rules, for instance based on the observation that any law of the form $x = f(y,z,\dots)$ was necessarily equivalent to the trivial law \eqref{eq2}, could already reduce the size of potential equivalence classes by a significant fraction. We discuss this method in \Cref{rewrite-sec}.
    \item The preorder axioms for $\vdash$, as well as the ``duality'' symmetry of the preorder with respect to replacing a magma operation $x \op y$ with its reflection $x \op^* y := y \op x$, can be used to significantly cut down on the number of implications that need to be proven explicitly; ultimately, only $10657$ ($0.05\%$) of the positive implications needed a direct proof.
    \item Automated Theorem Provers (ATP) could be deployed at acceptable compute cost to establish this generating set of positive implications, as discussed in \Cref{automated-sec}.
\end{itemize}

More challenging were the $13855357$ ($62.88\%$) implications that were false, $E \not \vdash E'$. Here, the range of techniques needed to refute such implications were quite varied.
\begin{itemize}
        \item Syntactic methods, such as observing an ``invariant'' of the law $E$ that was not shared by the law $E'$, could be used to obtain some refutations.  For instance, if both sides of $E$ had the same order, but both sides of $E'$ did not, this could be used to syntactically refute $E \vdash E'$.  Similarly, if the law $E$ was confluent, enjoyed a complete rewriting system, or otherwise permitted some understanding of the free magma associated to that law, one could decide the assertions $E \vdash E'$ for all possible laws $E'$, or at least a significant fraction of such laws.  We discuss these methods, and the extent to which they can be automated in \Cref{syntactic-sec}.  One novel such invariant we introduce here is the ``twisting semigroup'' of an equational theory, which gave simple refutations of some otherwise very challenging implications.
        \item Small finite magmas, which can be described explicitly by multiplication tables, could be tested by brute force computations to provide a large number of counterexamples to implications, or by ATP-assisted methods. See \Cref{finite-sec}.
        \item Linear models, in which the magma operation took the form $x \op y = ax + by$ for some (commuting or non-commuting) coefficients $a,b$, allowed for another large class of counterexamples to implications, which could be automatically scanned for either by brute force or by Grobner basis type calculations. See \Cref{linear-sec}.
        \item Translation invariant models, in which the magma operation took the form $x \op y = x + f(y-x)$ on an additive group, or $x \op y = x f(x^{-1} y)$ on a non-commutative group, reduce matters to analyzing certain functional equations; see ???.
        \item Greedy methods, in which either the multiplication table $(x,y) \mapsto x \op y$ or the function $f$ determining a translation-invariant model are iteratively constructed by a greedy algorithm subject to a well-chosen ruleset, were effective in resolving many implications not easily disposed of by preceding methods. See \Cref{greedy-sec}.
        \item Starting with a simple base magma $M$ obeying both $E$ and $E'$, and either enlarging it to a larger magma $M' \supset M$, extending it to a magma $N$ with a projection homomorphism $\pi: N \to M$, or modifying the multiplication table on a small number of values, also proved effective when combined with greedy methods. See \Cref{modify-base}.
        \item Some \emph{ad hoc} models based on existing mathematical objects, such as infinite trees, rings of polynomials, or ``Kisielewicz models'' utilizing the prime factorization of the natural numbers, could also handle some otherwise difficult cases.  In some cases, the magma law induced some relevant and familiar structures, such as a directed graph or a partial order, which also helped guide counterexample constructions. See \Cref{adhoc-sec}.
        \item Automated theorem provers were helpful in identifying which simplifying axioms could be added to the magma without jeopardizing the ability to refute the desired implication $E \vdash E'$.
\end{itemize}

    \note{Mention other outcomes, e.g. finite magma graph}
